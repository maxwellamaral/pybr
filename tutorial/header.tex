% Configuração para renderizar emojis no PDF
% Usar com: pandoc arquivo.md -o arquivo.pdf --pdf-engine=xelatex -H header.tex

% Definir fontes que suportam Unicode e emojis
\usepackage{fontspec}
\usepackage{xunicode}

% Configuração de capa personalizada
\usepackage{titling}
\usepackage{graphicx}

% Redefinir a página de título para uma capa mais bonita
\renewcommand{\maketitlehooka}{%
  \begin{center}
  \vspace*{2cm}
}
\renewcommand{\maketitlehookb}{%
  \vspace{1cm}
}
\renewcommand{\maketitlehookc}{%
  \vspace{1cm}
  \Large
  \textbf{$subtitle$}\\[0.5cm]
  \large
  $description$
  \vspace{2cm}
  \end{center}
}
\renewcommand{\maketitlehookd}{%
  \vfill
  \begin{center}
  \large
  $footer_text$\\
  \texttt{https://github.com/maxwellamaral/pybr}
  \end{center}
  \vspace{1cm}
}

% Fonte principal do documento
\setmainfont{DejaVu Serif}

% Fonte para código/monospace
\setmonofont{DejaVu Sans Mono}

% Configurar fallback para caracteres especiais (emojis)
% No Windows: Segoe UI Emoji
% No Mac: Apple Color Emoji  
% No Linux: Noto Color Emoji
\newfontfamily\emojifont{Segoe UI Emoji}[Scale=MatchLowercase]

% Redefinir comandos para usar a fonte de emoji quando necessário
\usepackage{newunicodechar}

% Configurar alguns emojis comuns
\newunicodechar{🚀}{\emojifont 🚀}
\newunicodechar{💻}{\emojifont 💻}
\newunicodechar{🪟}{\emojifont 🪟}
\newunicodechar{🍎}{\emojifont 🍎}
\newunicodechar{🐧}{\emojifont 🐧}
\newunicodechar{📝}{\emojifont 📝}
\newunicodechar{💡}{\emojifont 💡}
\newunicodechar{❓}{\emojifont ❓}
\newunicodechar{🎓}{\emojifont 🎓}
\newunicodechar{🎉}{\emojifont 🎉}
\newunicodechar{🐍}{\emojifont 🐍}
\newunicodechar{⚠}{\emojifont ⚠}
\newunicodechar{✅}{\emojifont ✅}
\newunicodechar{🎯}{\emojifont 🎯}
\newunicodechar{📖}{\emojifont 📖}
\newunicodechar{🔑}{\emojifont 🔑}
\newunicodechar{📦}{\emojifont 📦}
\newunicodechar{🔢}{\emojifont 🔢}
\newunicodechar{💾}{\emojifont 💾}
\newunicodechar{📁}{\emojifont 📁}
\newunicodechar{✓}{\emojifont ✓}
\newunicodechar{✗}{\emojifont ✗}
\newunicodechar{📈}{\emojifont 📈}
\newunicodechar{📉}{\emojifont 📉}
\newunicodechar{🏆}{\emojifont 🏆}
\newunicodechar{😊}{\emojifont 😊}
\newunicodechar{😐}{\emojifont 😐}
\newunicodechar{😔}{\emojifont 😔}
\newunicodechar{❌}{\emojifont ❌}
\newunicodechar{📚}{\emojifont 📚}
\newunicodechar{⚡}{\emojifont ⚡}
\newunicodechar{✨}{\emojifont ✨}
\newunicodechar{📊}{\emojifont 📊}

% Configurações adicionais para melhor renderização
\usepackage{hyperref}
\hypersetup{
    colorlinks=true,
    linkcolor=blue,
    filecolor=magenta,      
    urlcolor=cyan,
    pdftitle={Tutorial PyBR},
    pdfauthor={PyBR},
}

% Melhorar espaçamento
\usepackage{setspace}
\setstretch{1.2}
